
\documentclass{article}
\usepackage{graphicx}

\graphicspath{ {C:\Users\inama\Documents\images} }

\title{Schwarmsimulation}
\date{20-03-2020}
\author{Daniel Inama, Philip Hopfinger}




\begin{document}
\maketitle
\newpage

\section{Was ist eine Schwarmsimulation?}
Eine Schwarmsimulation besteht aus einzelen objekten, auch Booids genannt. Diese Boids folgen drei Regeln: 
\begin{itemize}
	\item Separation: Steuere weg von einer ansammlung an Boids
	\item Angleichung: wähle eine Richtung, welche der anderen Boids entspricht
	\item Zusammenhalt: bewege dich in Richtung mittlerer Postion der anderen Boids
\end{itemize}
\begin{figure}
	\includegraphics[width=\linewidth]{separation.gif}
	\caption{separation}
\label{}{separation}
\end{figure}


Figure \ref{fig:separation} shows a boat.

\end{document}
